\chap {Programovací jazyk C}
Jazyk C je jedním z nejstarších a nejvlivnějších programovacích jazyků v historii vývoje softwaru, který byl vyvinut v 70. letech 20. století v Bellových laboratořích. Díky svým vlastnostem, jednoduchosti a výkonnosti si získal široké uplatnění a stal se základem pro mnoho moderních programovacích jazyků, jako je C++, JavaScript a dokonce i některé jazyky vyšších úrovní jako Java a Python. 

\sec {Vlastnosti jazyka C}
\begitems
* {\bf Nízká úroveň abstrakce} - Jazyk C poskytuje abstrakci na použitým hardwarem (na úrovni jazyka programátora nezajímá jaký procesor programuje), ale zároveň umožňuje přímí přístup k paměti. 
* {\bf Vysoká optimalizace} - Díky své jednoduchosti je možné využít vysokou míru optimalizace výsledného programu. Díky tomu poskytují výsledné programy vysoký výkon.
* {\bf Jednoduchost a přímočarost} - Syntaxe jazyka C je jednoduchá a přímočará (v porovnání s jazyky vyšší úrovně). 
\enditems

\sec {Použití jazyka C}
\begitems
* {\bf Systémové programování} -  C je široce používán pro vývoj operačních systémů (například UNIX), ovladačů zařízení a dalších nízkoúrovňových aplikací, kde je vyžadován přímý přístup k hardwaru.
* {\bf Aplikace s vysokým výkonem} - Vzhledem k tomu, že C poskytuje velkou kontrolu nad výkonem a pamětí, je ideální pro vývoj aplikací, kde jsou kladeny vysoké nároky na výkon, jako jsou grafické programy, hry, simulace nebo vědecké výpočty.
* {\bf Embedded systémy} - C je jazykem první volby pro vývoj embedded (vestavěných) systémů, kde je efektivní využívání paměti a výkonu klíčové. Příkladem vestavěných systémů je například chytrý domácí spotřebič (robotický vysavač, pračka, ...)
\enditems

