\chap {Příprava prostředí pro vývoj v jazyce C}
Prostředí pro programování v jazyce C je možné připravit na jakémkoli počítači s pomocí nástrojů, které jsou zdarma ke stažení z internetu. V základu se jedná o kompilátor {\bf gcc}, který má za úkol převést zdrojový kód na spustitelný soubor a nástroj pro usnadnění a automatizaci překladu {\bf make}. 

\sec {Příprava prostředí na MS Windows}
Na systémech MS Windows je nejjednodušší způsob pro nastavení prostředí pro vývoj v jazyce C použít program {\bf msys2} (\url {https://www.msys2.org/}). Tento program vytvoří strukturu nástrojů, které jsou k dispozici na systémech Linux. Po instalaci je třeba nejprve třeba nastavit systém, aby mohl vyhledat nástroje, které se do počítače nainstalovali. K tomu je nutné upravit {\bf proměnnou prostředí PATH}, do které je třeba přidat dvě nové cesty. Proměnné prostředí je možně upravovat zadáním příkazu do nabídky start: {\it upravit proměnné prostředí systému}

\vskip 5mm
\picw=.7\hsize \centerline{\inspic {\imgpath setup_environment_variables.png} }\nobreak\medskip
\caption/f Nastavení proměnné prostředí Path

Poté se otevře okno pro nastavení proměnných prostředí ve kterém je tlačítko {\bf Proměnné prostředí}, které otevře okno s proměnnými prostředí, které jsou v systému vytvořené. V sekce {\bf Uživatelské proměnné} se nachází výčet proměnných prostředí pro aktuálně přihlášeného uživatele a mezi nimi se nachází proměnné {\bf Path}. Po jejím označení je třeba stisknout tlačítko upravit. Následně by se mělo otevřít okno pro úpravu hodnot v této proměnné. Nyní je nutné přidat dvě nové cesty pomocí tlačítka {\bf Nový} a nastavit jejich hodnotu na: 
\begtt 
C:\msys2\usr\bin 
\endtt 
 
\begtt 
C:\msys2\mingw64\bin 
\endtt 

Následně stačí jen vše uložit stiskem tlačítka OK a cesty k nástrojům systému msys2 by měly být nastavené.

\vskip 5mm
\picw=.9\hsize \centerline{\inspic {\imgpath environment_variables.png} }\nobreak\medskip
\caption/f Okno pro nastavení proměnných prostředí

Nově nastavené cesty je možné věřit otevřením příkazové řádky pomocí zadání příkazu {\it cmd} do nabídky start a zadat příkaz, který by měl provést aktualizaci systému msys2:
\begtt
$ pacman -Syu
\endtt

(Prosím nekopírujte příkaz se znakem \$, ten značí že se jedná o terminálový vstup)\par
V případě, že jsou cesty správně nastavené by se měl spustit balíčkový správce {\it pacman}, který slouží k instalaci a aktualizaci programů do prostředí msys2 a požádá vás o potvrzení spuštění instalace aktualizací:

\vskip 5mm
\picw=.9\hsize \centerline{\inspic {\imgpath msys2_update.png} }\nobreak\medskip
\caption/f Okno pro nastavení proměnných prostředí

Aktualizace a jiné instalace se potvrdí prostým stiskem klávesy {\it Enter}.

Po aktualizaci prostředí msys2 je potřeba nainstalovat kompilátor pro jazyk C {\bf gcc}, který má za úkol převést zdrojový k na spustitelný a nástroj pro automatizaci překladu {\bf make}. To se provede zadáním příkazu:
\begtt
$ pacman -S gcc make
\endtt

Některé knihonvy, které jsou použity při programování jsou uloženy na specifické cestě v systému. Ta se může lišit v závislosti na použitém systému (MS Windows s prostředím msys2, GNU Linux, Mac OS). Z tohoto důvodu se využívá pro vyhledání cest k instalovaným knihovnám nástroj {\bf pkg-config}. Nástroj pkg-config využívá malé soubory, které osahují konfiguraci pro knihovnu instalovanou v lokálním systému, které mají příponu .pc a jsou uloženy v konfiguračním adresaři. Na systému msys2, ale nástroj pkg-config neví kde má tyto soubory hledat a je nutné mu je nastavit. K tomu slouží proměnná prostředí {\bf PKG\_CONFIG\_PATH}, kterou je nutné nastavit stejně jako proměnnou prostředí path v předchozím bodě. V tomto případě, jsou konfigurační soubory uloženy v adresáři: 
\begtt
C:\msys2\ming64\lib\pkgconfig
\endtt

\sec {Příprava prostředí na GNU Linux}
Na systémech Linux bývá často nástroj gcc a make dostupný již v základní instalaci. Z tohoto důvodu je příprava prostředí na vývoj v jazyce C na systémech Linux snazší než na jiných systémech. Pokud nástroje gcc a make není v základní instalaci, záleží na distribuci instalovaného Linuxového systému. 

Na linuxové distribuci Ubuntu je možné nástroje gcc a make instalovat buď prostřednictvím nástroje {\bf Ubuntu Software Center} a nebo pomocí příkazové rádky:
\begtt
$ sudo apt install gcc make
\endtt

\sec {Příprava prostředí na Mac OS}

\sec {IDE}
Poté co jsou v systému nainstalované nástroje gcc a make je prostředí pro vývoj programů vjazyce C připraveno k použití, ale je potřeba nainstalovat tzv. IDE (Integrated development environment), zjednodušeně řečeno textový editor, který obsahuje nástroje pro zjednodušení psaní zdrojových kódu. Nejjednodušší IDE je {\bf Visual studio code} (\url {https://code.visualstudio.com/}). Jeho instalace je jednoduchá a přímočará. 

\vskip 5mm
\picw=.9\hsize \centerline{\inspic {\imgpath vs_code.png} }\nobreak\medskip
\caption/f Visual studio code


